\documentclass[11pt, a4paper]{article}

%%%%%%%%%%%%%%%%%
% Configuration %
%%%%%%%%%%%%%%%%%
\usepackage{allrunes}
\usepackage{amsmath}
% If magyar is wanted
% \usepackage[magyar]{babel}
\usepackage[T1]{fontenc}
\usepackage[utf8]{inputenc}
\usepackage{fixltx2e}
\usepackage{multirow}
\usepackage{url}
\usepackage{amsfonts}
\usepackage{amsthm}
\usepackage{mathtools}
\usepackage{amssymb}
\usepackage{xcolor}

% using circled symbols
\usepackage{tikz}
\newcommand*\circled[1]{\tikz[baseline=(char.base)]{
            \node[shape=circle,draw,inner sep=2pt] (char) {#1}}}


% Here you can configure the layout
\usepackage{geometry}
\geometry{top=1cm, bottom=1cm, left=1.25cm,right=1.25cm, includehead, includefoot}
\setlength{\columnsep}{7mm} % Column separation width

\usepackage{graphicx}

%\usepackage{gensymb}
\usepackage{float}

% For bra-ket notation
\usepackage{braket}

% To have a good appendix
\usepackage[toc,page]{appendix}

\usepackage{abstract}
\renewcommand{\abstractnamefont}{\normalfont\bfseries}
\renewcommand{\abstracttextfont}{\normalfont\small\itshape}
\usepackage{lipsum}

%%%%%%%%%%%%%%%%%%%
% Custom commands %
%%%%%%%%%%%%%%%%%%%
\newcommand{\bb}[1]{\mathbf{#1}}
\newcommand{\dd}{\mathrm{d}}
\newcommand{\Tr}[1]{\mathrm{Tr}\left[#1\right]}
\newcommand{\Sp}[1]{\mathrm{Sp}\left[{#1}\right]}

% \newtheorem*{tetel*}{Tétel}
% \newtheorem*{defn*}{Definíció}
% \newtheorem*{pld*}{Példa}
% \newtheorem*{megj*}{Megjegyzés}
% \newtheorem*{allit*}{Állítás}

% \newtheorem{tetel}{Tétel}
% \newtheorem{defn}{Definíció}
% \newtheorem{pld}{Példa}
% \newtheorem{megj}{Megjegyzés}
% \newtheorem{allit}{Állítás}

% Hyperref should be generally the last package to load
% Any configuration that should be done before the end of the preamble:

\usepackage{hyperref}
\hypersetup{colorlinks=true, urlcolor=blue, linkcolor=blue, citecolor=blue}

\title{Infocom networks cheat sheet}

\author{Dániel Nagy$^\dagger$}
\date{%
    $^\dagger$Department of Physics of Complex Systems, Eötvös Loránd University, H-1117, Pázmány Péter sétány 1/A. Budapest, Hungary\\[2ex]%
    \today
}

\begin{document}
\maketitle
\begin{abstract}
    I am too retarded.
\end{abstract}
\newpage
\begin{itemize}
    \item Exam topics:
    \begin{itemize}
        \item History, operation, measurement and topology of the Internet.
        \item Models of Internet network topology.
        \item Robustness of the Internet against random and deliberate attacks.
        \item Virtual and social networks over the Internet.
        \item Search, diffusion and epidemics in the Internet.
        \item Modeling Internet packet traffic.
    \end{itemize}
    \item The Internet itself is a network of heterogeneous networks mutually interconnected.
    \item Local Area Networks (buildings, university departments, etc.), Metropolitan Area Networks, Wide Area Networks (connect computers which are scattered over wide geographical areas).
    \item Routers choose the best routing for data. The router handles the packet by looking at the destination address and sending it to the neighboring router. The way the router decides which is the next hop router is determined by the routing protocol algorithms
    \item The Internet is built on a whole family of cooperative protocols often referred to as the Internet protocol suite
    \item The Internet is a packet switched network (all the communications between two hosts are mixed together with everyone else’s data, put in common pipes, delivered to the specified destination address)
    \item Transmission Control Protocol and Internet Protocol
    \item The main role of the TCP is to break the information down into packets of small and manageable size, stamped with the origin and destination IP.
    \item These protocols ensure that if a connection is broken, the system will find another path and send the packet to the destination.
    \item The path on which a packet will be going through is mostly unpredictable
    \item The Internet can be partitioned into autonomously administered domains, called autonomous systems (AS), which vary in size, geographical extent, and function. 
    \item There are Transit ASs and Stub ASs (leaves).
    \item The Internet can be modeled as an undirected graph where each edge represents a physical connection and each node represents a router.
    \item \textbf{MEASURES of the internet graph}:
    \item Vertices can be characterized by:
    \begin{itemize}
        \item $l_{ij}$ shortest path length
        \item $b_i$ betweenness of defines the total number of shortest paths among pairs of vertices in the network that pass through that vertex.
        \begin{equation*}
            b_i = \frac{\sum\limits_{j\neq i \neq k} \sigma_{jk}(i)}{\sigma_jk},
        \end{equation*}
        where $\sigma_{jk}(i)$ is the number of shortest paths between $j$ and $k$ that pass through $i$ and $\sigma_jk$ is the total number of shortest paths between $j$ and $k$.
        \item $c_i$ Clustering coefficient of the vertex $i$ is defined as the ratio between the number of edges $e_i$ among its nearest neighbors and its maximum possible value $k_i (k_i-1)/2$:
        \begin{equation*}
            c_i = \frac{2e_i}{k_i(k_i-1)}
        \end{equation*}
    \end{itemize}
    \item The network itself can be characterized by the averages
    \begin{equation*}
        \langle k\rangle, \langle c_i\rangle, \langle l_{ij}\rangle, \langle b_i \rangle
    \end{equation*}
    \item The average shortest path length $\langle l_{ij}\rangle$ among vertices found in Internet maps is very small if compared with the size of the graphs.
    \item The distribution $P_l(l)$ is sharply peaked around $\langle l \rangle \approx 15$.
    \item The small separation among Internet routers and ASs is a striking example of the so-called small-world effect.
    \item The small-world effect goes along with a high level of clustering.
    \item The degree distribution $p(k) \propto k^{-\gamma}$, $\gamma = 2.1$. Power-law with $2 < \gamma < 3 \implies$ scale-free network.
    \item The cumulative degree distribution $P^c(k) = \int\limits_k^{\infty}P(k')dk' = a\exp{-(k/k_c)^{\beta}}$, $k_c$ is the cutoff degree.
    \item This means that a $k \rightarrow \lambda k$ transformation does not change the distribution.
    \item The level of heterogeneity of networks can be characterized by
    \begin{equation*}
        \kappa = \frac{\langle k^2\rangle}{\langle k\rangle}.
    \end{equation*}
    \item For scale free networks $\kappa >> \langle k \rangle$.
    \item The betweenness distribution $P_b(b) \propto b^{-\gamma_b}$ with $\gamma_b \approx 2$. The cumuative distribution $P_b^c(b) = \int\limits_b^{\infty}P_b(b')db'\propto b^{1-\gamma_b}$
    \item The hierarchical nature of the Internet is represented by the tendency of high degree nodes to be well interconnected among each other. \textit{Rich club phenomenon}.
    \item Further analysis can be made by inspecting the degree correlation function $P(k'|k)$ which is the probability that a vertex with degree $k$ is 
    connected to a vertex with degree $k'$.
\end{itemize}


\end{document}